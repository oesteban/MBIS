\section{Software description}\label{sec:software}

\subsection{Existing software}
\label{sec:related_work}

Many fully automated brain tissue segmentation tools, based on 
  Bayesian classifiers, are readily available and widely used.
In \autoref{table:tools_comparison}, we present a comparison among
  representative existing tools, along with a brief summary of the
  unique features of each.
All the tools make use of the \gls*{multivariate_gaussian} model
  with \gls*{markov} regularization.
The tools listed in the table are \gls*{fast} \citep{zhang_segmentation_2001}, 
  \gls*{spm}, \gls*{leemput_ems}, \emph{ATROPOS} \citep{avants_open_2011},
  \emph{NiftySeg} \citep{cardoso_niftyseg:_2012}, \emph{Freesurfer} 
  \citep{fischl_freesurfer_2012},
  and the software proposed in the present study (\gls*{mbis}).
The presented tools generally share a base design that follows
  the flowchart in \autoref{fig:em_flowchart}.
It is important to note that Freesurfer and \gls*{spm} are not just
  segmentation utilities, but fully automated pipelines for brain
  \gls*{mri} processing and analysis that include brain tissue segmentation.
\Gls*{spm} provides an isolated interface (called \texttt{segment}) for the
  problem at hand, the methodology of which is described elsewhere
  \citep{ashburner_unified_2005}.
Conversely, Freesurfer provides precise \emph{hard} segmentations of the brain
  in a large number of individual neuroanatomical regions \citep{fischl_whole_2002},
  which can be appropriately fused to the three-tissue problem.
The features presented in \autoref{table:tools_comparison} regarding Freesurfer
  and \gls*{spm} refer only to their whole-brain segmentation processes.

\begin{table*}[!t]
\begin{minipage}[t]{\linewidth}
\caption[Brain tissue segmentation tools]{\label{table:tools_comparison}}
\rowcolors{2}{white}{lightgray}
\footnotesize
% @Author: oesteban
% @Date:   2014-03-26 15:40:27
% @Last Modified by:   oesteban
% @Last Modified time: 2014-03-26 15:55:12
\begin{tabularx}{\textwidth}{l|ZZZZZZ||Z}
\hline
              & \acrshort*{fast} & \acrshort*{spm} & \acrshort*{leemput_ems}     & ATROPOS    & NiftySeg & Freesurfer & \gls*{mbis} \\
\hline
%               FSL          SPM        EMS           ATROPOS      NIFTYSEG      FS    MBIS
Multivariate  & Partial    & Partial$^{1}$ & Full       & Full      &  Full      & No  & Full \\
Optimization  & \acrshort*{icm}  & \acrshort*{icm} & \acrshort*{montecarlo}& \acrshort*{icm}  & Unknown   & \acrshort*{icm} & \acrshort*{graph_cuts} \\
Bias model    & Polynomial & \acrshort*{dct} & Polynomial & No$^{3}$  & Unknown & No$^{3}$ & B-spline \\
Atlas usage   & Available  & Intensive & Available  & Available &  Intensive & Intensive & Available \\
License       & GPL        & GPL       & BSD-like   & BSD       &  BSD       & Freeware & GPL \\
Platform      & Unix       & Matlab    & SPM8       & Any$^{2}$       &  Any$^{2}$       & Unix & Any$^{2}$ \\
Reference     & \citep{zhang_segmentation_2001} & \citep{ashburner_unified_2005} & \citep{van_leemput_unifying_2003} & \citep{avants_open_2011} & \citep{cardoso_niftyseg:_2012} & 
\citep{fischl_whole_2002} & \\
\hline
\end{tabularx}
\footnotesize
$^{1}$ Work in progress. \\
$^{2}$ \emph{Any} platform supported by the \emph{CMake} building system.\\
$^{3}$ The tool does not integrate a bias model, but it is released along with an external tool for correction.\\
\end{minipage}
\end{table*}

The first feature to be compared is multivariate implementation.
\Gls*{leemput_ems}, ATROPOS and NiftySeg fully support the
\gls*{multivariate_gaussian} model.
\Acrshort*{spm} is currently integrating support for multivariate
  data, while \gls*{fast} provides multichannel segmentation that
  importantly differs from the univariate segmentation methodology.
Freesurfer only supports \gls*{t1} \gls*{mri} as input for segmentation.

The model estimation is always performed with the \gls*{e_m} algorithm,
  possibly with some improvements.
For instance, \gls*{leemput_ems} implements a
  robust estimator and \gls*{pv} constraints.
Therefore, this property has been omitted in \autoref{table:tools_comparison}.
The main differences are found in the \gls*{markov} energy minimization 
  problem, \gls*{icm} being the most used methodology.
\Gls*{leemput_ems} implements \acrfull*{montecarlo} sampling, which is more
  reliable than \gls*{icm} but computationally expensive.
\Gls*{mbis} is the first tool among the surveyed software packages to include
  \gls*{graph_cuts} optimization, for which a great trade-off
  between efficiency and correctness has been proven.

Another important feature is the bias field correction, generally solved by
  approximation of linear combinations of smooth basis functions.
\Gls*{fast} and \gls*{leemput_ems} use polynomial least-squares fitting,
  \gls*{spm} uses the \gls*{dct} with \gls*{lm} optimization, and \gls*{mbis}
  uses B-splines basis.
Unfortunately, there was no information available about the bias model
  implemented in NiftySeg at the time of writing.
Two of the surveyed tools do not internally integrate a bias model:
Freesurfer provides a pipeline including a previous correction utility,
  and ATROPOS advises the prior use N4ITK \citep{tustison_n4itk:_2010}.

The next point of comparison is the use of atlases to initialize the algorithm and/or
  to aid the estimation of model parameters.
All the tools can initialize segmentation using prior atlas information.
Those tools that also use priors throughout the model fitting are denoted
  with ``intensive'' atlas use in \autoref{table:tools_comparison}.

In terms of software availability, for all the tools the source
  code is publicly released and the software is distributed under 
  open-source licenses.
With respect to their installation, a number of them (\gls*{fast}, \gls*{spm} 
  and \gls*{leemput_ems}) are platform-dependent, whereas the others
  are multi-platform using the \emph{CMake} building tool (\url{http://www.cmake.org}).


\subsection{Design considerations}\label{sec:design}
Given the described context of existing software, we aimed to design a
  multivariate segmentation tool, which is flexible, easy to use, comprehensive,
  and would also include a \gls*{graph_cuts} solver and a B-spline bias field model.
As a result we designed \gls*{mbis}, an open-source and cross-platform software that
  supports multivariate data by design and that integrates all the methods described in
  \autoref{sec:implementation_details}.
Segmentation provided with \gls*{mbis} is general purpose.
In this study, \gls*{mbis} is specifically adapted to the 3D
  brain tissue segmentation problem.
In order to facilitate contributions by third-party developers, the code
  follows the standards of \gls*{itk}, and some interfaces have been defined
  to integrate new code, preserving the software modularity.

We also promote \emph{reproducible research}, a concept that is drawing increasing
  interest in parallel to the proliferation of computational solutions
  for image processing problems.
Following the definition of \citeauthor{vandewalle_reproducible_2009}
  \citep{vandewalle_reproducible_2009}, we release here an
  open-source bundle with evaluation experiments based on open data
  to help the community replicate and test our work \citep{yoo_open_2005,
  ibanez_open_2006}.
  
In order to evaluate \gls*{mbis} comprehensively, we define three
  validation targets.
Consistently with the design considerations mentioned above,
  we test the performance of \gls*{mbis} with three different
  open data resources containing multivariate and multi-site
  data.
Full details of these databases are provided in \autoref{table:data} 
  (\ref{a:data}), describing the \gls*{mri} sequences involved and
  their specific parameters.
Finally, we address these targets in three different experiments
  (the results are presented in \autoref{sec:results}).

The first experiment evaluates the accuracy of segmentation, with 
  comparison to \gls*{fast}, using one simulated dataset.
The evaluation framework includes tests to calibrate the best
  parameters for the tool, benchmarks of the bias field estimation,
  and studies the impact of spatial misalignment between channels.
The second experiment evaluates the reproducibility of results, in
  similar settings to a recent validation study \citep{de_boer_accuracy_2010}.
Unfortunately, the database used by \citeauthor{de_boer_accuracy_2010} is not
  publicly available.
Hence, the resulting figures are not directly comparable to their work as
  we used different data.
On one hand, we studied the repeatability of the segmentation by analyzing
  the differences in tissue volumes.
On the other hand, the overlap indices described in \autoref{sec:evaluation_indices}
  were evaluated.
The third section of the evaluation framework is an exemplary pipeline of
  tissue volume analysis in large-scale databases.
We illustrated the use of \gls*{mbis} on such applications, segmenting
  multivariate \gls*{mri} datasets from 584 healthy subjects, and correlating
  tissue volumes with age.

\subsection{Evaluation framework}
\label{sec:experimental_framework}
The evaluation framework is built using \gls*{nipype}, in order to facilitate the
  fulfillment of the requirements of reproducible research.
The evaluation includes a \gls*{nipype} \emph{Interface} to \gls*{mbis},
  three \gls*{nipype} \emph{Workflows} to implement the experiments
  described in \autoref{sec:design} and a set of scripts in Python to automate the
  execution of the workflows and presentation of results (figures and tables
  included in this paper).
To assess and compare results appropriately in terms of accuracy and 
  robustness \citep{altman_measurement_1983}, we evaluate two families of
  indicators: volume agreements and overlap indices.
  
\paragraph{Volume agreement}
Volume agreement between the segmentation found and the ground-truth, or
  between segmentations of corresponding time points, is a commonly used
  benchmark.
Volumes of the identified tissues are directly related to the total size
  of the brain.
Therefore, we provide here the ``\gls*{icv} fraction'' of each tissue
  as the ratio of the measured volume over the total volume of
  the whole-brain.

\paragraph{Overlap indices\label{sec:evaluation_indices}}
Overlap is a widely used indicator to assess segmentation results
  with respect to a ground-truth \citep{crum_generalized_2006}.
We use a \gls*{ev_fsi} derived from the fuzzy \gls*{ev_ji} \citep{crum_generalized_2006},
  as in eq. \eqref{eq:si_index}.
This fuzzy index definition takes the resulting \glspl*{tpm} as inputs,
  and naturally extends the binary definition.
We refer to the binary index as the \gls*{ev_si}, when computed on
  \emph{hard} segmentations.
Generally, the definition of \gls*{ev_ji} tends to favor classes with greater volume
  when computing the average overlap of several classes.
Thus, results reporting averages of overlap indices are compensated
  for tissue volume in this paper.
Additional indices are also provided:
  \gls*{ev_tpf} that acts as a measure of sensitivity \eqref{eq:tpf_index};
  \gls*{ev_ef} that expresses the over-segmentation \eqref{eq:ef_index};
  and \gls*{ev_oc}, which is reported as an alternative to the 
  \gls*{ev_si} \eqref{eq:oc_index}.
\begin{align}
fSI,SI&=\frac{2\,JI}{1+JI}\,, \label{eq:si_index} \\
TPF&=\frac{TP}{TP+FN}\,, \label{eq:tpf_index} \\
EF&=\frac{FP}{TP+FN}\,, \label{eq:ef_index} \\
OC&=1-\frac{FP+FN}{TP}\,, \label{eq:oc_index}
\end{align}
where $TP$ stands for true positives, $FP$ for false positives, and $FN$ for false 
negatives. We did not extend these measures to the probabilistic results, so they
are only used for the assessment of the \emph{hard} segmentations.
