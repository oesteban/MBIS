\section{Introduction}
\label{sec:introduction}
%
Brain tissue segmentation from \gls*{mri} has been one of
  the most challenging problems in computer vision
  applied to biomedical image analysis \citep{kapur_segmentation_1996}.
It is intended to provide precise delineations of \gls*{wm}, \gls*{gm}
  and \gls*{csf} from acquired data.
Brain tissue segmentation is the standpoint of processing schemes in an
  endless number of research studies concerning brain morphology,
  such as quantitative analyses of tissue volumes 
  \citep{mortamet_effects_2005,abe_sex_2010,taki_correlations_2011},
  studies of cortical thickness \citep{fischl_measuring_2000,jones_three-dimensional_2000,
  macdonald_automated_2000}, and voxel-based morphometry
  \citep{wright_voxel-based_1995,paus_structural_1999,
  good_voxel-based_2001,ge_age-related_2002}.
In a clinical context, numerous studies have demonstrated the
  potential use of brain tissue segmentation.
The spatial location of the above key anatomical structures within
  the brain is a requirement for clinical intervention \citep{kikinis_digital_1996} 
  (e.g. radiotherapy planning, surgical planning, and image-guided intervention).
Early applications addressed global conditions; for example \citep{tanabe_tissue_1997}
  used semiautomated segmentation of \gls*{mri} to assess the decrease in 
  total brain tissue and cortical \gls*{gm}, and ventricle enlargement in Alzheimer's
  Disease patients.
Another study \citep{hazlett_cortical_2006} presented an automated methodology
  to identify abnormal increase of the \gls*{gm} volume in individuals
  with autism.
Focal conditions have also been studied, including extra classes in clustering and
  some other adaptations of methods to pathologies, such as automated tumor delineation
  \citep{prastawa_automatic_2003}, lesion detection and volume analyses
  in multiple sclerosis \citep{collins_automated_2001,
  zijdenbos_automatic_1998,zijdenbos_automatic_2002,van_leemput_automated_2001,
  van_leemput_unifying_2003}, and white matter lesions associated
  with age and several conditions like clinically silent stroke, and higher
  systolic blood pressure \citep{anbeek_probabilistic_2004}.
The accurate and automated segmentation of tumor and
  edema in multivariate brain images is an active field of interest in medical
  image analysis, as illustrated by the Challenge on \emph{Multimodal
  Brain Tumor Segmentation} \citep{menze_multimodal_2014} that has been 
  held in conjunction with the last three sessions of the 
  \gls*{miccai} International Conference.

A survey on brain tissue segmentation techniques is reported elsewhere 
  \citep{liew_current_2006}.
Currently popular methodologies can be grouped into three main
  families.
\emph{Deformable model fitting} approaches
  \citep{suri_leaking_2000,yushkevich_user-guided_2006,roura_marga:_2012,delibasis_novel_2013,dang_validation_2013}
  are designed to evolve a number of initial contours
  towards the intensity steps that occur at tissue
  interfaces.
\emph{Atlas-based methods} \citep{gorthi_active_2011}
  use image registration
  to perform a spatial mapping between the actual data
  and an anatomical reference called an atlas.
The atlas is \emph{prior} knowledge on the morphology
  of data, and it generally comprehends a partition previously 
  extracted by any other means (i.e. manual delineation,
  averaging large populations, etc.).
\emph{Clustering or classification} algorithms
  \citep{van_leemput_automated_1999-1,ahmed_modified_2002,
  vrooman_multi-spectral_2007,ji_generalized_2012}
  search for a pixel-wise partition of the image 
  data into a certain number of categories or clusters
  (i.e. \gls*{wm}, \gls*{gm}, and \gls*{csf}).
The partition can be \emph{hard} when each pixel belongs
  to a single cluster or \emph{fuzzy},
  assigning a probability of membership to
  each category, which yields a so-called \gls*{tpm} 
  per class.
These three families of segmentation strategies have often been
  combined to obtain enhanced results.
For instance, deformable models can be initialized
  using contours already located close to the solution 
  sought using atlases.
In clustering methods, \emph{priors} usually take the
  form of precomputed \glspl*{tpm} derived from the atlas.
These prior probability maps can be used just to initialize
  the model, or be integrated throughout the model fitting
  process \citep{ashburner_unified_2005}, simultaneously
  improving the atlas registration at each iteration.
The use of \emph{priors} presents two particular properties.
On one hand, it generally aids the segmentation
  process providing great stability and robustness.
However, it is also suspected to bias results,
  driving the solution somewhat close to the
  population features that underlie the atlas
  \citep{davatzikos_why_2004}.
One further concern about the use of \emph{priors} is
  posed by the need for a spatial mapping of the atlas
  information to the actual data 
  \citep{bookstein_voxel-based_2001,
  ashburner_why_2001}, typically performed through a
  registration process that may not be trivial or
  flawless \citep{crum_zen_2003}.
The unpredictable morphology found in pathologic brains
  discourages the use of atlases extracted from healthy populations.
Conversely, monospectral and strictly data-driven approaches are
  usually very unreliable for pathologic subjects.
For instance, a previous study \citep{prastawa_automatic_2003} updated a
  standard atlas with an approximation of tumor locations for automated
  clustering-based segmentation.
On the other hand, multivariate approaches with outlier detection
  \citep{van_leemput_automated_2001} have been proposed in the
  case of multiple sclerosis derived lesions.

The tool proposed in this work, named \gls*{mbis}, belongs to the sub-group of
  Bayesian classification methods, which have been successfully
  applied to brain tissue segmentation for the last 20 years
  \citep{van_leemput_automated_1999-1}.
Therefore, we will restrict the scope of this paper to this
  sub-group of clustering methods.
Given the maturity of the field, numerous evaluation studies have been reported
  \citep{cuadra_comparison_2005,de_boer_accuracy_2010,roche_convergence_2011},
  along with further refinements or extensions to the original methodologies
  \citep{zhang_segmentation_2001,van_leemput_unifying_2003,
  ashburner_unified_2005,fischl_whole_2002}.
Existing applications of brain tissue segmentation generally
  use \gls*{mri} as input data as a safe, noninvasive,
  and highly precise modality.
Early applications typically selected \acrfull*{t1}
  MPRAGE sequences, mainly for their particularly appropriate
  contrast between soft tissues, and for their wide availability.
The current clinical setup provides a large number
  of different sequences that can be used to
  characterize each voxel of the brain with a vector of
  intensities from each different \gls*{mri} scheme.
In the last decade, we have witnessed an
  explosion of the number of \gls*{mri} sequences widely available, enabling
  the exploration of new observed features and requiring powerful multivariate
  processing and analysis.
Moreover, the vast amount of multi-site data
  that research and clinical routines produce daily,
  necessitates accurate and robust methods to
  perform fully automated segmentation on heterogeneous 
  (in the sense of multi-centric and/or multi-scanner) data
  reliably.

In this paper, we contribute to the field with \gls*{mbis}, an open-source 
  software suite to perform multivariate segmentation on 
  heterogeneous data.
We also present a comprehensive evaluation framework,
  containing several validation experiments on data
  from three publicly-available resources.
The first experiment demonstrates the accuracy of \gls*{mbis} segmenting
  one synthetic dataset, in comparison to \gls*{fast}, a widely-used tool.
The second experiment demonstrates the repeatability of results,
  reporting the disagreement between segmentations of two
  multivariate images of the same subject.
These images correspond to 21 subjects who underwent a scan-rescan
  session with the same \gls*{mri} protocol acquired twice.
The third experiment proves the suitability of \gls*{mbis} on large-scale
  segmentation studies.
We demonstrate the successful application of \gls*{mbis} on a multi-site
  resource of 584 subjects and observe the aging effects over
  tissue volumes.

The manuscript is structured as follows: In \autoref{sec:methods},
  after introducing the theoretical background, we describe the
  particular features of the method implemented by \gls*{mbis},
  highlighting its methodological novelties.
In \autoref{sec:software}, we review the existing software that can be
  used to perform brain tissue segmentation, and compare it to \gls*{mbis}.
We also present the design considerations that underlie
  this work, and we describe the evaluation framework.
In \autoref{sec:results}, we describe the specific details of each
  experiment, illustrating the usefulness of \gls*{mbis}
  and reporting the results of evaluation.
Finally, we discuss in \autoref{sec:discussion} the three 
  experiments, and envision the unique opportunity that
  multivariate segmentation of the latest \gls*{mri} sequences
  provides.